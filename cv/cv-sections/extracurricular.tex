%----------------------------------------------------------------------------------------
%	SECTION TITLE
%----------------------------------------------------------------------------------------

\cvsection{Extracurricular}
\cventry
{Langauge: Java}
{Telegram Contest}
{}
{}
{
Telegram official coding competition for Android, iOS and JS developers. \linebreak
The goal is to develop software for showing simple charts based on input data telegram provide.\linebreak
All the submitted code written by own from scratch.\linebreak
Git repo: \url{https://github.com/sysint64/telegram-chart-contest}\linebreak
}

\cventry
{Langauge: Kotlin}
{Intellij-DLanguage}
{}
{}
{
Support for the D Programming Language within IntelliJ IDEA\linebreak
Source code: \url{https://github.com/intellij-dlanguage/intellij-dlanguage}\linebreak
Contributions: \url{https://github.com/intellij-dlanguage/intellij-dlanguage/pulls?q=is:pr+is:closed+author:sysint64}\linebreak
}

\cventry
{Langauge: D}
{Phobos}
{}
{}
{
Phobos is the standard library that comes with the D Programming Language Compiler.\linebreak
Source code: \url{https://github.com/dlang/phobos}\linebreak
Contributions: \url{https://github.com/dlang/phobos/pulls?q=is:pr+is:closed+author:sysint64}
}

\cvsection{Personal projects}

%----------------------------------------------------------------------------------------
%	SECTION CONTENT
%----------------------------------------------------------------------------------------

\begin{cventries}

%------------------------------------------------

\cventry
{Langauge: D}
{Simple Data Oriented OpenGL GAPI}
{}
{}
{
Fast, simple and cross platform data oriented OpenGL GAPI library for D;\linebreak
Source code: \url{https://github.com/sysint64/gapi}\linebreak
Examples: \url{https://github.com/sysint64/gapi-examples}\linebreak
}

\cventry
{Langauge: Kotlin, Python with Django}
{Vocabulator}
{}
{}
{
Android application for compiling a personal dictionary which can helps you improve your vocabulary in English, Japanesse or any other languages.\linebreak
Android application source code: \url{https://github.com/sysint64/vocabulator-android-client}\linebreak
Server side source code: \url{https://github.com/sysint64/vocabulator-server-side}\linebreak
}

\cventry
{Langauge: D}
{RPUI}
{}
{}
{
Fast, simple and cross platform graphical interface library based on OpenGL and using RPDL as mark up language
and for configurations like shortkeys, settings etc.\linebreak
Source code: \url{https://github.com/sysint64/RPUI}
}

\cventry
{Langauge: D}
{RPDL}
{}
{}
{
Simple declarative language written on D with compile time loading and bytecode compilation.\linebreak
Source code: \url{https://github.com/sysint64/RPDL}
}

\cventry
{Langauge: D}
{dapt}
{}
{}
{
D attribute processor (like java annotation processor) designed for automatic types and attributes collection and generate
code based on the information has collected.\linebreak
Source code: \url{https://github.com/sysint64/dapt}\linebreak
Examples: \url{https://github.com/sysint64/dapt-examples}
}

\cventry
{Language: Django with Python}
{Ripa Archive}
{}
{}
{
Documents organizer, have interface familiar with OS file manager for manipulate folders and documents, set statuses
for documents, display log activity and etc.\linebreak
Souce code: \url{https://github.com/sysint64/ripa_archive}
}

\cventry
{Langauges: C++, D} % Affiliation/role
{E2DIT} % Organization/group
{} % Location
{Dec. 2014 - PRESENT} % Date(s)
{ % Description(s) of experience/contributions/knowledge
E2DIT - 2D map editor for games with rich functionality for manipulation of objects and change their tology. \linebreak
Application contain own UI Toolkit written on OpenGL and own format for serialization which can compiled to bytecode (e2ml). \linebreak
\linebreak
\begin{cvitems}
    \item {Rewriting bad architecture solution and begin using C++14: \url{https://github.com/sysint64/e2dit}}
    \item {First unfinished implementation: \url{https://bitbucket.org/lveteam/lve-mapeditor-framework}}
\end{cvitems}
}

\cventry
{Langauge: C++} % Affiliation/role
{APC Language} % Organization/group
{} % Location
{Jun. 2012 - 2014} % Date(s)
{ % Description(s) of experience/contributions/knowledge
Implemented new language with static typing, classes, SIMD vectors etc. \linebreak
Source code with examples: \url{https://github.com/sysint64/programming-language-APC}
}

%% \cventry
%% {Langauges: Delphi} % Affiliation/role
%% {3D Lightmapping} % Organization/group
%% {} % Location
%% {2011} % Date(s)
%% { % Description(s) of experience/contributions/knowledge
%% 3D Lightmapping algorithm implementation. Source code: \url{https://github.com/sysint64/lightmapping-3d}
%% }

\cventry
{Langauges: Delphi} % Affiliation/role
{Tetris} % Organization/group
{} % Location
{2011} % Date(s)
{ % Description(s) of experience/contributions/knowledge
Classic tetris with additional modes. Source code: \url{https://github.com/sysint64/tetris-glscene}
}

\cventry
{Langauges: Delphi} % Affiliation/role
{Fun Smile} % Organization/group
{} % Location
{2010} % Date(s)
{ % Description(s) of experience/contributions/knowledge
2D Arcade game. Source code: \url{https://github.com/sysint64/funsmile}
}

%------------------------------------------------

\end{cventries}
